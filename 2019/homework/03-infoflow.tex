\documentclass[10pt]{article}
\usepackage{amsmath,amssymb,fullpage,graphicx}
\let\hat\widehat
\let\tilde\widetilde

\usepackage{xcolor}
\usepackage{stmaryrd}
\usepackage{mathpartir}

\usepackage{verbatim}
\usepackage{hyperref}
\usepackage{listings}
\usepackage{graphicx}
\usepackage{mathabx}
%\usepackage{lsyntax}
\newtheorem{theorem}{Theorem}
\newtheorem{lemma}[theorem]{Lemma}
\newtheorem{example}[theorem]{Example}
\newtheorem{corollary}[theorem]{Corollary}
\newtheorem{proposition}[theorem]{Proposition}
\newtheorem{remark}[theorem]{Remark}
\newenvironment{proof}{\textbf{ Proof.}}{$\Box$}

\newcommand{\answer}[1]{\text{{\it #1}}}
\newcommand{\solution}[1]{\textbf{Solution:} #1}
%\newcommand{\solution}[1]{}

\usepackage[irlabel]{bugcatch}
% \usepackage[bracketinterpret,seqinfers,sidenotecalculus]{logic}
% \newcommand{\I}{\interpretation[const=I]}

% \newcommand{\bebecomes}{\mathrel{::=}}
% \newcommand{\alternative}{~|~}
% \newcommand{\asfml}{F}
% \newcommand{\bsfml}{G}
% \newcommand{\cusfml}{C}
% \def\leftrule{L}%
% \def\rightrule{R}%

%%%% new version of enumerate with less spacing
\newenvironment{enum}{
\begin{enumerate}
  \setlength{\itemsep}{1pt}
  \setlength{\parskip}{0pt}
  \setlength{\parsep}{0pt}
}{\end{enumerate}}

\parskip 10pt
\parindent 0pt
\pagenumbering{gobble}
\newcommand{\note}[1]{\ \\{\small\color{red}\emph{#1}}\\}

\lstset{%
  basicstyle=\ttfamily\small,
  keywordstyle=\bfseries\color{green!60!black},
  commentstyle=\itshape\color{purple!60!black},
  stringstyle=\color{orange}
}

\newcommand{\pand}[2]{\keywordfont{and}(#1, #2)}
\newcommand{\por}[2]{\keywordfont{or}(#1, #2)}

\begin{document}

\begin{center}
\textbf{ Assignment 3: The Highs and Lows of Information Flow\\15-316 Software Foundations of Security and Privacy}\\
\end{center}
Due: \textbf{ 11:59pm}, Wednesday 10/30/19 \\
Total Points: 50

\vspace{-5mm}

\begin{enumerate}
\item \textbf{Flow types (15 points).} 

Consider the following program.
\begin{lstlisting}[escapechar=\#]
      if(a = b) {
        c := 0
        d := d + 1
      } else {
        d := c #$\times$# e
      }
      b := c
\end{lstlisting}
\paragraph{Part 1 (5 points).}
Identify a \emph{minimal} policy $\Gamma$ under which this program type checks in the information flow type system described in lecture.
The policy that you form must assign $\Gamma(a) = \hisec$, and be minimal in the sense that it assigns as few variables the label $\hisec$ as possible while still type checking.

\paragraph{Part 2 (10 points).}
Use the rules of the information flow type system to show that the program typechecks under your policy.

%%
%% Write your solution here
%%

% \newpage

\item \textbf{Exclusive interference (20 points).}

Consider the following program under the policy $\Gamma = (a : \hisec, b : \hisec, c : \lowsec)$.

\begin{lstlisting}[escapechar=\#]
      if(a > 0) {
        if(b > 0) {
          c := 0;
        } else {
          c := 1;
        }
      } else {
        if(b > 0) {
          c := 1;
        } else {
          c := 0;
        }
      }
\end{lstlisting}

\paragraph{Part 1 (5 points).}
Show that this program does not satisfy noninterference by providing a pair of inputs $(a, b, c)$ and $(a', b', c')$ that violate the formal definition given in lecture.

\paragraph{Part 2 (10 points).}
Although this program does not satisfy noninterference, does it leak any information about the $\hisec$ variables $a$ and $b$ to an observer who sees the initial and final values of $c$?
Describe the feasible set of initial values of $a, b$ to justify your answer.

\paragraph{Part 3 (5 points).}
Building on the insights gained in the previous parts of this question, suppose that we propose a declassification rule for exclusive-or terms.
(\irref{declassxor}) below says that when $\astrm, \bstrm$ take Boolean (0, 1) values, then their exclusive-or can safely be leaked to the bottom security label.
\[
\cinferenceRule[declassxor|DeclassXor]{Declassify Xor}
{\linferenceRule[sequent]
  {\lsequent{\Gamma}{\astrm : \ell_1}
  &\lsequent{\Gamma}{\bstrm : \ell_2}
  &\astrm, \bstrm \text{ evaluate to either } \{0,1\}}
  {\lsequent{\Gamma}{\astrm \oplus \bstrm : \bot}}
}{}%
\]
Explain why this may be a justifiable rule to use in terms of the uncertainty remaining about $\astrm,\bstrm$.
You may reference points that you have already made in Part 2 of this question.


%%
%% Write your solution here
%%

% \newpage

\item \textbf{Leveraging interference (15 points).} 

Consider the following program, under the type context $\Gamma = (a : \hisec, b : \lowsec, c : \lowsec)$.
\begin{lstlisting}[escapechar=\#]
      if(a < 0) {
        if(b < a) c := 0
        else c := 1
      } else {
        if(a < b) c := 0
        else c := 1
      }
\end{lstlisting}
Describe a procedure that leverages the fact that this program does not satisfy non-interference under $\Gamma$ to learn the value of the $\hisec$-typed variable. You can make use of the following assumptions.
\begin{itemize}
\item Assume that an attacker can control the values of $\lowsec$-typed variables prior to executing the program, and observe their value afterwards. They can neither control nor observe $\hisec$ variables at any point. 
\item $-N \le a \le N$ for some $N$ whose value is unknown to the attacker. 
\item Finally, the attacker can run the program with different $\lowsec$ inputs any number of times, and the $\hisec$ input will remain the same.
\end{itemize}
How many times does the attacker need to run the program using your procedure to learn $a$? 

%%
%% Write your solution here
%%

\end{enumerate}

\end{document}
