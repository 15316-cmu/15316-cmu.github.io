\documentclass[11pt]{article}

\usepackage[margin=1.0in]{geometry}
\usepackage{lecnotes}
% logic macros, to be filled from
% lmacros-f21.tex (15-814, F21) and lmacros-f16 (15816, F16)

\newcommand{\ddd}{\raisebox{0.2em}[1.1em]{$\vdots$}}
\newcommand{\vctr}[1]{\begin{array}[c]{c}#1\end{array}}
\newcommand{\CC}{\mathcal{C}}
\newcommand{\DD}{\mathcal{D}}
\newcommand{\EE}{\mathcal{E}}
\newcommand{\FF}{\mathcal{F}}

\newcommand{\ms}[1]{\mathsf{#1}}
\newcommand{\mi}[1]{\mathit{#1}}
\newcommand{\mb}[1]{\mathbf{#1}}
\newcommand{\mt}[1]{\mathtt{#1}}
\newcommand{\ums}[1]{\underline{\ms{#1}}}
\newcommand{\oms}[1]{\overline{\ms{#1}}}
\newcommand{\mc}[1]{\mathcal{#1}}

\newcommand{\uscore}{\mbox{\tt\char`\_}}
\newcommand{\arrow}{\mathbin{\rightarrow}}

\newcommand{\semi}{\mathrel{;}}

% from stmaryrd
\newcommand{\lbb}{\llbracket}
\newcommand{\rbb}{\rrbracket}

% evaluation
\newcommand{\eval}{\ms{eval}}
\newcommand{\evalz}{\ms{eval}_{\mathbb{Z}}}
\newcommand{\evalb}{\ms{eval}_{\mathbb{B}}}

% authorization logic
\newcommand{\says}{\mathrel{\mb{says}}}
\newcommand{\aff}{\mathrel{\mb{aff}}}
\newcommand{\llet}[3]{\mb{let}\; #1 = #2\; \mb{in}\; #3}
\newcommand{\alet}[4]{\mb{let}\; \{#1\}_{#2} = #3\; \mb{in}\; #4}

% \newcommand{\rep}[1]{{}^\ulcorner\! #1{}^\urcorner}

\newcommand{\new}[1]{{\color{blue}#1}}




\usepackage{framed}

\newcommand{\course}{15-316: Software Foundations of Security \& Privacy}
\newcommand{\lecturer}{}
\newcommand{\lecdate}{Due Tue, Feb 10, 2026}
\newcommand{\lecnum}{3}
\newcommand{\lectitle}{Homework 3 (Solutions Template)}
\newcommand{\courseurl}{https://15316-cmu.github.io/2026/}

\title{Assignment \lecnum \\ Proving Safety \\ Solutions Template}
\date{\lecdate \\ 75 points}

\lhead[\fancyplain{}{\bfseries A\lecnum.\thepage}]%
      {\fancyplain{}{\bfseries\lectitle}}
\chead[]{}
\rhead[\fancyplain{}{\bfseries\lectitle}]%
      {\fancyplain{}{\bfseries A\lecnum.\thepage}}
\lfoot[{\small\scshape Assignments}]{{\small\scshape Assignments}}
\cfoot[]{}
\rfoot[{\small\scshape\lecdate}]{{\small\scshape\lecdate}}

\begin{document}

\maketitle

\noindent
This file is a template for writing up your solutions.

Your solution should be handed in as a file \texttt{hw3.pdf} to Gradescope.  If
at all possible, write your solutions in \LaTeX.  The handout
\texttt{hw3-safety.zip} includes the \LaTeX\ sources for Lectures 5 and 6 and
the necessary style files which provide some examples for rules, derivations,
and proofs.

\section{Semantic Equivalence [20 points]}

In the previous homework we proved program equivalence using the inference rules
of the sequent calculus for dynamic logic.  For programs that use $\mb{while}$
this is difficult or impossible with the tools we have so far, since our axiom
for $\mb{while}$ is only an implication (not a bi-implication) and,
consequently, only corresponds to a right rule ($[\mb{while}]R$) in the sequent
calculus and not a left rule.

\begin{task}[20 points]\rm
  Consider the two formulas
  \[
    [\mb{while}\; \top\; \mb{assert} \top] Q
  \]
  and
  \[
    [\mb{test}\; \bot] Q
  \]
  Prove that they are \emph{semantically} equivalent, that is, in any state
  $\omega$, one is true if and only if the other is.  For the reasons explained
  above, you should do this using the definition of $\omega \models P$ and other
  semantic definitions such as $\omega \lbb \alpha \rbb \nu$ and
  $\omega \lbb \alpha \rbb \lightning$.
\end{task}

\begin{framed}
% Write your solution here.
\end{framed}

\clearpage
\section{Safety of Output [55 points]}

In this problem we consider a more complex safety policy than division by zero
or array access out of bounds.  We add three new commands to our language: one
to open a stream for writing, one to print to a stream, and one to close a
stream.  To keep complexity manageable, we imagine there is just one stream
(e.g., the terminal) that we can open, print to, and close:
\[
  \begin{array}{llcl}
    \mbox{Programs} & \alpha, \beta & ::= & \ldots \mid \mb{open} \mid \mb{print}\; e \mid \mb{close}
  \end{array}
\]
In this problem we assume this is the only source of unsafe behavior, that is,
we exclude division and memory access from consideration.

As an example, here is a program that prints the numbers 0 to 9.
\begin{tabbing}
\qquad \= $\mb{open} \semi$ \\
\> $i := 0 \semi$ \\
\> $\mb{while}\; (i < 10)\; \{$ \\
\> \quad \= $\mb{print}\; i \semi$ \\
\>\> $i := i+1$ \\
\> $\} \semi$ \\
\> $\mb{close}$
\end{tabbing}
We have the following safety conditions:
\begin{enumerate}
\item \label{item:pre} When a program starts to execute, we should assume the stream is closed.
\item \label{item:post} When a program finishes, we should verify that the stream is closed.
\item Only when the stream is closed can we open it.
\item Only when the stream is open can we print to it. 
\item Only when the stream is open can we close it.
\end{enumerate}
In order to define the meaning of these programs, we introduce a new \emph{ghost
  variable} ``$\mi{status}$'' that is used to track the status of the stream.
It is called a ghost variable because it is only introduced to reason about the
safety of the program and may not appear directly in a source program.  When
$\mi{status} = 0$ the stream is not open, and when $\mi{status} \neq 0$ then the
stream is open.

\begin{task}[5 pts]\rm
  The sample program above should be safe.  State each of the following (by
  necessity using the ghost variable $\mi{status}$):
  \begin{enumerate}[(a)]
  \item The precondition required for its safety (see \autoref{item:pre})
  \item The postcondition required for its safety (see \autoref{item:post})
  \item The loop invariant needed to prove safety
  \end{enumerate}
\end{task}

\begin{framed}
% Write your solution here.
\end{framed}

In order to model the output of a program, the state now \emph{should} contain a
sequence of values that have been printed.  However, since we are only
interested in safety we are content to \emph{approximate} the true meaning of a
program and ignore the values actually printed.

\begin{task}[15 pts]\rm
  \label{task:semantics}
  Give definitions for $\omega \lbb \alpha \rbb \nu$ and
  $\omega \lbb \alpha \rbb \lightning$ for each of the three new constructs.

  Make sure to model the prescribed safety conditions accurately.  In other
  words $\omega \lbb \alpha \rbb \lightning$ should be true exactly when the
  program is unsafe, and no unsafe program should have a poststate.
\end{task}

\begin{framed}
% Write your solution here.
\end{framed}

\begin{task}[15 pts]\rm
  Give \emph{right} rules in the sequent calculus for $\mb{open}$,
  $\mb{print}$, and $\mb{close}$.

  You do not need to show left rules or axioms, and you do not need to prove
  their soundness.
\end{task}

\begin{framed}
% Write your solution here.
\end{framed}

\begin{task}[5 pts]\rm
  Give a biconditional (iff) axiom for reasoning about $\mb{print}\; e$
  of the form
  \[
    [\mb{print}\; e]Q \leftrightarrow {??}
  \]
\end{task}

\begin{framed}
% Write your solution here.
\end{framed}

\begin{task}[10 pts]\rm
  Prove the validity of your axiom with respect to the semantics in
  \autoref{task:semantics}.
\end{task}

\begin{framed}
% Write your solution here.
\end{framed}

Consider the following program schema (with $P(i)$ and $R(i)$ some formulas of
pure arithmetic that may refer to $i$):

\begin{tabbing}
  $\mb{while}\; P(i)\; \{$ \\
  \quad $\mb{if}\; R(i)\; \mb{then}\; \mb{close}\; \mb{else}\; \mb{open} \semi$ \\
  \quad $\mb{print}\; i \semi$ \\
  \quad $\mb{if}\; R(i)\; \mb{then}\; \mb{open}\; \mb{else}\; \mb{close} \semi$ \\
  \quad $i := i + 1$ \\
  $\}$
\end{tabbing}

\begin{task}[5 pts]\rm
  Assume suitable pre- and post-conditions to satisfy \autoref{item:pre} and
  \autoref{item:post} in the informal definition of safety.

  State conditions relating $P(i)$ and $R(i)$ that would guarantee safety of
  this program.  They should be as general as possible (yes, we realize that
  $P(i) = \bot$ guarantees safety regardless of $R(i)$).
\end{task}

\begin{framed}
% Write your solution here.
\end{framed}

% \bibliographystyle{alpha}
% \bibliography{fp,lfs}

\end{document}
