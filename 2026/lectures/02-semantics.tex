\documentclass[11pt]{article}

\usepackage[margin=1.5in]{geometry}
\usepackage{lecnotes}
% \usepackage{mpass}
% \lstset{style=verb,language=mpass}
% logic macros, to be filled from
% lmacros-f21.tex (15-814, F21) and lmacros-f16 (15816, F16)

\newcommand{\ddd}{\raisebox{0.2em}[1.1em]{$\vdots$}}
\newcommand{\vctr}[1]{\begin{array}[c]{c}#1\end{array}}
\newcommand{\CC}{\mathcal{C}}
\newcommand{\DD}{\mathcal{D}}
\newcommand{\EE}{\mathcal{E}}
\newcommand{\FF}{\mathcal{F}}

\newcommand{\ms}[1]{\mathsf{#1}}
\newcommand{\mi}[1]{\mathit{#1}}
\newcommand{\mb}[1]{\mathbf{#1}}
\newcommand{\mt}[1]{\mathtt{#1}}
\newcommand{\ums}[1]{\underline{\ms{#1}}}
\newcommand{\oms}[1]{\overline{\ms{#1}}}
\newcommand{\mc}[1]{\mathcal{#1}}

\newcommand{\uscore}{\mbox{\tt\char`\_}}
\newcommand{\arrow}{\mathbin{\rightarrow}}

\newcommand{\semi}{\mathrel{;}}

% from stmaryrd
\newcommand{\lbb}{\llbracket}
\newcommand{\rbb}{\rrbracket}

% evaluation
\newcommand{\eval}{\ms{eval}}
\newcommand{\evalz}{\ms{eval}_{\mathbb{Z}}}
\newcommand{\evalb}{\ms{eval}_{\mathbb{B}}}

% authorization logic
\newcommand{\says}{\mathrel{\mb{says}}}
\newcommand{\aff}{\mathrel{\mb{aff}}}
\newcommand{\llet}[3]{\mb{let}\; #1 = #2\; \mb{in}\; #3}
\newcommand{\alet}[4]{\mb{let}\; \{#1\}_{#2} = #3\; \mb{in}\; #4}

% \newcommand{\rep}[1]{{}^\ulcorner\! #1{}^\urcorner}

\newcommand{\new}[1]{{\color{blue}#1}}





\let\m\mb

\newcommand{\course}{15-316: Software Foundations of Security \& Privacy}
\newcommand{\lecturer}{Matt Fredrikson}
\newcommand{\lecdate}{January 15, 2026}
\newcommand{\lecnum}{2}
\newcommand{\lectitle}{Safety \& Semantics}
\newcommand{\courseurl}{https://15316-cmu.github.io/2026/}

\begin{document}

\maketitle

\section{Introduction}

We begin by talking about safety, and in particular one important class of safety
properties: memory safety. Mistakes that programmers make having to do with memory
safety have been responsible for a lot of security vulnerabilities over the years,
and we will discuss some of the different types of memory safety errors that lead
to vulnerabilities in common programs.

Our goal is to write code that does not have these vulnerabilities. There are
many ways that have been proposed to avoid memory safety errors: type systems
that ensure that memory is always used correctly, tools that check for memory
safety errors at compile time, runtime systems that detect and prevent errors,
and many others at various parts of the software and hardware stack. To understand
how these work, and what their limitations and tradeoffs are, we need to understand
how memory safety materializes in programs at the level of precise semantics.

In the second part of this lecture, we will go into detail to understand the
semantics of programs and how they relate to proving things about how programs
will behave when run. We won't yet define exactly what memory safety means
at this level, but we will have laid the essential groundwork to do this,
as well as to start reasoning about the safety of programs in more systematic
ways.

\section{Safety \& Liveness}

Our goal in this course is to reason about security properties of programs so we
can prevent vulnerabilities and fend off attacks.  There are different kinds of
such properties, and they require different techniques to enforce them.  One way
to classify them is to think about them as properties of \emph{traces} of a
program, that is, the (possibly infinite) sequence of states or events that take
place when a program executes.
\begin{description}
\item[Safety Properties] Intuitively, a safety property means that
  ``\emph{nothing bad happens}'' during a computation.  So every finite prefix
  of a trace should satisfy some specification that excludes ``bad'' states or
  events.  Common examples of ``bad'' are programs that, in C, have undefined
  behavior.  This includes division by zero, integer overflow, double free, or
  accessing memory whose value is undefined.  The latter is exploited in
  so-called \emph{buffer overflow attacks}.  An example from concurrency are
  race conditions between threads.  Another example is a policy that requires
  that a principal is authorized before giving them access to a resource, in
  which case the ``bad'' thing is unauthorized access.
\item[Liveness Properties] Intuitively, a liveness property captures that
  ``\emph{something good happens}'' during an execution.  For example, a server
  should eventually respond to a request, or a deleted file should actually
  disappear and be no longer recoverable.
\end{description}
Liveness properties are more intrinsically more difficult to reason about and
enforce than safety properties.
There are also security properties that can not be captured as properties of a
single trace.  We will consider such a property, namely \emph{information flow},
in the second part of the course. For now, we'll work towards an understanding
of a type of safety property, memory safety, that plays an important role in
secure coding.

\subsection{Memory Safety}

Informally, a program is \emph{memory safe} if every memory access it performs is
well-defined: it only reads from and writes to memory that it is allowed to
access, and it interprets that memory consistently with the way it was allocated
and initialized. In low-level languages like C and C++, violations of memory
safety often fall into ``undefined behavior'', meaning that the language does not
prescribe what happens next. This is where security comes into the picture,
because it is often possible to shape what happens next into something useful
for an attacker.

We'll refer to memory safety \emph{vulnerabilities} as the security-relevant
instances of memory safety errors: situations where an attacker can influence
program behavior or observe secrets by triggering an invalid memory access. Let's
take a quick look at some of the ways that this can arise in programs.

\begin{description}
\item[Out-of-bounds] Out-of-bounds vulnerabilities are a very broad class of
  errors that occur when a program reads or writes outside the bounds of an
  allocated object. They commonly arise from unsafe array indexing, pointer
  arithmetic errors, or integer overflows that miscompute sizes or offsets.
  There are numerous ways that they can be exploited, from control hijacking
  (e.g. via overwritten control data such as return addresses or function
  pointers) to reading sensitive data from neighboring memory.
\item[Use-after-free] Use-after-free vulnerabilities arise, as the name
  suggests, when a program continues to access memory after it has been
  deallocated. They typically result from confusion about ownership patterns or
  aliasing where one part of the program frees an object while another still
  holds a pointer to it. A typical exploitation pattern for UAF errors involve
  forcing the freed memory to be reallocated with attacker-controlled data, so
  that subsequent accesses operate on maliciously crafted contents.
\item[Uninitialized memory use] When a program reads memory that has been
  allocated but not fully initialized with defined values, then stack variables,
  partially initialized structs, or reused heap allocations may leak sensitive
  information or lead to control hijacking via attacker-controlled residual
  data. Although compilers, libraries, and widely-used runtime environments have
  done a lot to mitigate the frequency and severity of these errors, they remain
  a true security problem.
\item[Race conditions] These vulnerabilities occur when concurrent threads or
  processes access and modify shared memory without proper synchronization,
  leading to inconsistent views of object state or lifetime. They often manifest
  as use-after-free or out-of-bounds accesses when one thread frees or resizes
  an object while another is using it. Attackers exploit timing windows to
  trigger invalid memory accesses in a controlled way. The consequences include
  memory corruption, data leakage, and arbitrary code execution.
\end{description}

Each of these types of memory safety errors can manifest in countless ways.
Our goal is to guarantee that none of them exist in programs that we write,
but first we will need to find a way to define what memory safety is, so that
we can be sure that we have addressed every possible case that falls into
the scope described in this section.

\section{Straight-Line Programs}

We now present our small imperative programming language in stages.
The development is inherently open-ended in the sense that we will
introduce more constructs as our study goes on.

For the sake of simplicity we assume that all
variables range of the integers $\mathbb{Z}$.  We have a simple
language of \emph{arithmetic expressions} $e$, with the usual
conventions that we do not detail here.  We use $a$, $b$, $c$ for
integer constants and $x$ to stand for variables.
\[
  \begin{array}{llcl}
    \mbox{Arithmetic Expressions} & e & ::=
    & c \mid x \mid e_1 + e_2 \mid e_1 - e_2 \mid e_1 * e_2 \mid \ldots
  \end{array}
\]
Since expressions contains variables, their meaning is determined with
respect to a \emph{state} that assigns integers to variables.  We use
$\omega$, $\mu$, $\nu$ to range over states and assume that they are
defined on all variables.  We write $\omega(x) = c$ if $\omega$ maps
$x$ to $c$.  Then the value of expression $e$ in state $\omega$ is
written as
\[
  \omega \lbb e\rbb = c
\]
and is easily defined based on the structure of $e$.  For example:
\[
  \begin{array}{lcl}
    \omega \lbb c\rbb & = & c \\
    \omega \lbb x\rbb & = & \omega(x) \\
    \omega \lbb e_1 + e_2\rbb & = & \omega\lbb e_1\rbb + \omega\lbb e_2\rbb \\
    \omega \lbb e_1 - e_2\rbb & = & \omega\lbb e_1\rbb - \omega\lbb e_2\rbb \\
    \ldots
  \end{array}
\]
The last two equations may look somewhat odd---we have to keep in mind
that `$+$' and `$-$' on the left-hand side are pieces of syntax that
form expressions while `$+$' and `$-$' on the right-hand side are the
mathematical operations on integers.  Other operations are defined
analogously.

Programs are denoted by $\alpha$ and $\beta$ and we start here with
two simple constructs: \emph{assignment} $x := e$ and
\emph{sequential composition} $\alpha \semi \beta$.
\[
  \begin{array}{llcl}
    \mbox{Programs} & \alpha, \beta & ::=
    & x := e \mid \alpha \semi \beta \mid \ldots
  \end{array}
\]
The meaning of a program is a \emph{relation} between the
\emph{prestate} and \emph{poststate} of its execution.  It is a
relation instead of a function because we would like to accommodate
nonterminating programs (no possible poststate) and also
nondeterministic programs (multiple possible poststates).  We write
\[
  \omega \lbb \alpha\rbb \nu
\]
if the meaning of the program $\alpha$ relates prestate $\omega$ to
poststate $\nu$.

We define the meaning of assignment $x := e$ to evaluate
$e$ in the current state to $c$ and then update the state to map $x$
to $c$.  In symbols:
\[
  \omega \lbb x := e\rbb \nu \quad\mbox{iff}\quad
  \mbox{$\nu = \omega[x \mapsto c]$ where $c = \omega\lbb e\rbb$}
\]
Here we use the notation $\omega[x \mapsto c]$ for the result of
updating the state $\omega$ by mapping $x$ to $c$ (no matter what it
was before).

The meaning of sequential composition $\alpha \semi \beta$ is to
execute first $\alpha$ and then $\beta$ from the resulting state.
That is:
\[
  \omega \lbb \alpha \semi \beta\rbb \nu
  \quad\mbox{iff}\quad
  \mbox{there is a $\mu$ such that $\omega \lbb \alpha \rbb \mu$
    and $\mu \lbb \beta \rbb \nu$}
\]
In other words, the relation denoted by $\alpha \semi \beta$ is the
\emph{composition} of the relations denoted by $\alpha$ and $\beta$.

As an example, let's compute
\[
  (\omega[x \mapsto a]) \lbb x := x + 2\rbb \nu
\]
and we find
$\nu = (\omega[x \mapsto a])[x\mapsto a+2] = \omega[x \mapsto a+2]$
Slightly more complicated is
\[
  (\omega[x \mapsto a]) \lbb x := x + 1 \semi x := x + 1\rbb \nu
\]
We determine that there is an intermediate state
$\mu = \omega[x \mapsto a+1]$ and a final state
$\nu = \omega[x \mapsto a+2]$.

So, both of these programs define the same relation between
$\omega[x\mapsto a]$ and $\omega[x \mapsto a+2]$ Therefore we can
state that these two programs are semantically equivalent
\[
  \lbb x := x+2\rbb = \lbb x := x+1 \semi x := x+1 \rbb
\]
They have the same meaning because they have the same effects
on the state.  This, by the way, might fail to be true if the language
were extended to allow shared memory concurrency because another
process can intervene after the first assignment on the right, while
the left atomically increments $x$ by two.  Lesson: we always have to
be careful about the extent of the language when we reason about it,
be it semantically (as here) or logically (as in the next lecture).

\section{Conditionals}

We now add conditionals $\m{if}\, P\, \alpha\, \beta$ to our language,
read as ``\emph{if $P$ then $\alpha$ else $\beta$}''.  A
characteristic of our setup is that formulas $P$ do double duty: on one
hand they serve as conditions in if-then-else programs and (shortly)
guards on while loops.  On the other hand we also use them to
\emph{reason} about programs as shown in the next lectures.
\[
  \begin{array}{llcl}
    \mbox{Programs} & \alpha, \beta & ::=
    & x := e \mid \alpha \semi \beta \mid \m{if}\, P\, \alpha\, \beta \mid \ldots \\
    \mbox{Formulas} & P, Q & ::=
    & e_1 = e_2 \mid e_1 \leq e_2 \mid \top \mid \bot \mid P \land Q \mid P \lor Q \\
                    & & \mid & P \rightarrow Q \mid \lnot P \mid \ldots
  \end{array}
\]
In concrete syntax we usually write $\top$ (top) as ``$\m{true}$'', $\bot$
(bottom) as ``$\m{false}$'', and $\lnot$ as ``$\m{not}$''.

In order to define the meaning of the conditional, we first need to
define the meaning of the formulas, in mathematical terms.  Because
variables range just over integers, the
language of formulas we are concerned with is that of \emph{integer
  arithmetic}.  We define their meaning relative to an assignment
$\omega$ of values to variables
\[
  \mbox{$\omega \models P$ \qquad \emph{$P$ is true in state $\omega$}}
\]
It is defined on the structure of $P$.
\[
  \begin{array}{ll}
    \omega \models \top & \mbox{always} \\
    \omega \models \bot & \mbox{never} \\
    \omega \models e_1 = e_2 & \mbox{iff $\omega\lbb e_1\rbb = \omega\lbb e_2\rbb$} \\
	    \omega \models e_1 \leq e_2 & \mbox{iff $\omega\lbb e_1\rbb \leq \omega\lbb e_2\rbb$} \\
	    \omega \models P \land Q & \mbox{iff $\omega \models P$ and $\omega \models Q$} \\
	    \omega \models P \lor Q & \mbox{iff $\omega \models P$ or $\omega \models Q$} \\
	    \omega \models \lnot P & \mbox{iff $\omega \not\models P$} \\
	    \omega \models P \rightarrow Q & \mbox{iff whenever $\omega \models P$ then also $\omega \models Q$}
	  \end{array}
\]
For the fragment of formulas above, the truth of a formula in a state is
straightforward to decide by evaluating expressions in that state and then
computing the Boolean connectives.

With this out of the way, we can now define the meaning of the conditional
by cases on the truth of $P$.
\[
  \begin{array}{lcl}
    \omega \lbb \m{if}\, P\, \alpha\, \beta \rbb \nu
    & \mbox{iff}
    & \mbox{$\omega \lbb \alpha\rbb \nu$ when $\omega \models P$} \\
    & & \mbox{$\omega \lbb \beta\rbb \nu$ when $\omega \not\models P$}
  \end{array}
\]

%%%%%%%%%%%%%%%%%%%%%%%%%%%%%%%%%%%%%%%%%%%%%%

\section{While Loops}

The abstract syntax for while loops is $\m{while}\, P\, \alpha$ which
should somehow be the same as
$\m{if}\, P\, (\alpha \semi \m{while}\, P\, \alpha)\, \m{skip}$, where
$\m{skip}$ is a program that has no effect. Although it is
perfectly possible to make this work as a so-called \emph{inductive
  definition}, it has the issue that $\m{while}\, P\, \alpha$ appears
on both sides.  So we break it down by ``guessing'' the number of
iterations of the loop, using an auxiliary relation $\lbb \m{while}\, P\, \alpha \rbb^n$
indexed by an $n \geq 0$.  If $n = 0$ we must exit the loop so $P$ should
be false, and if $n > 0$ we should go around the loop once, followed by $n-1$
more iterations.
\[
  \begin{array}{lcl}
    \omega \lbb \m{while}\, P\, \alpha\rbb \nu
    & \mbox{iff} &
    \begin{array}[t]{l}
      \omega \lbb \m{while}\, P\,\alpha\rbb^n \nu \\
      \mbox{for some $n \geq 0$}
    \end{array}
                   \\[1ex]
    \omega \lbb \m{while}\, P\, \alpha\rbb^0 \nu 
    & \mbox{iff} & \mbox{$\omega \not\models P$ and $\omega = \nu$} \\
    \omega \lbb \m{while}\, P\, \alpha\rbb^{n+1} \nu
    & \mbox{iff} &
    \begin{array}[t]{l}
      \mbox{$\omega \models P$ and $\omega \lbb \alpha\rbb \mu$ for some $\mu$} \\
      \mbox{and $\mu \lbb \m{while}\, P\, \alpha\rbb^n \nu$}
    \end{array}
    \\
  \end{array}
\]


We can appeal to this definition to compute the meaning of a few
simple programs.  Actually, we will look at whole families of programs
because it doesn't matter what some of the components are.  For
example, any program $\m{while}\; \m{false}\; \alpha$ will behave the
same, regardless of $\alpha$.  Instead of looking up the answer
immediately, we suggest solving these yourself first with careful
reference to the definitions.
\[
  \begin{array}{ll}
    \omega\lbb \m{while}\, \m{true}\, \alpha\rbb \nu \\
    \omega\lbb \m{while}\, \m{false}\, \alpha\rbb \nu \\
    \omega\lbb x := x\rbb \nu \\
  \end{array}
\]
% You can find the answers on the next page.
% \clearpage
We calculate
\[
  \begin{array}{ll}
    \omega\lbb \m{while}\, \m{true}\, \alpha\rbb \nu & \mbox{never} \\
    \omega\lbb \m{while}\, \m{false}\, \alpha\rbb \nu & \mbox{iff $\nu = \omega$} \\
    \omega\lbb x := x\rbb \nu & \mbox{iff $\nu = \omega$}
  \end{array}
\]
We see, for example, that
\[
  \lbb \m{while}\, \m{false}\, \alpha\rbb = \lbb x := x\rbb
\]
where the equality here denotes an equality between two
relations. Further examples in the next section.

\section{Summary}

\begin{figure}[ht]
  \[
    \begin{array}{lcl}
      \omega \lbb c\rbb & = & c \\
      \omega \lbb x\rbb & = & \omega(x) \\
      \omega \lbb e_1 + e_2\rbb & = & \omega \lbb e_1\rbb + \omega \lbb e_2\rbb \\
      \omega \lbb e_1 * e_2\rbb & = & \omega \lbb e_1\rbb \times \omega \lbb e_2\rbb
    \end{array}
  \]
  \caption{Semantics of Expressions}
  \label{fig:exps}
\end{figure}

\begin{figure}[ht]
  \[
    \begin{array}{lcl}
      \omega\lbb x := e\rbb \nu
      & \mbox{iff}
      & \omega[x \mapsto c] = \nu \;\mbox{where}\; \omega \lbb e\rbb = c
      \\[1ex]
      \omega\lbb \alpha \semi \beta\rbb \nu
      & \mbox{iff}
      & \omega\lbb \alpha\rbb \mu \;\mbox{and}\; \mu\lbb \beta \rbb \nu\; \mbox{for some state $\mu$}
      \\[1ex]
      \omega \lbb \mb{if}\; P\; \mb{then}\; \alpha\; \mb{else}\; \beta \rbb \nu
      & \mbox{iff}
      & \omega \models P\; \mbox{and}\; \omega\lbb \alpha\rbb \nu \quad \mbox{or} \\
      & & \omega \not\models P\; \mbox{and}\; \omega \lbb \beta\rbb \nu
      \\[1ex]
      \omega \lbb \mb{while}\; P\; \alpha\rbb \nu
      & \mbox{iff} & \omega \lbb \mb{while}\; P\; \alpha\rbb^n\nu \quad \mbox{for some $n \in \mathbb{N}$} \\[1em]
      \omega \lbb \mb{while}\; P\; \alpha\rbb^{n+1} \nu
      & \mbox{iff} & \omega \models P \;\mbox{and}\; \omega\lbb \alpha\rbb \mu\;
                     \mbox{and}\; \mu \lbb \mb{while}\; P\; \alpha\rbb^n \nu \\
      \omega \lbb \mb{while}\; P\; \alpha\rbb^0 \nu
      & \mbox{iff} & \omega \not\models P \;\mbox{and}\; \omega = \nu
    \end{array}
  \]
  \caption{Semantics of Programs}
  \label{fig:progs}
\end{figure}

\begin{figure}[ht]
  \[
    \begin{array}{lcl}
      \omega \models e_1 \leq e_2 & \mbox{iff} & \omega \lbb e_1\rbb \leq \omega \lbb e_2\rbb \\
      \omega \models e_1 = e_2 & \mbox{iff} & \omega \lbb e_1\rbb = \omega \lbb e_2\rbb
      \\[1em]
      \omega \models P \land Q & \mbox{iff} & \omega \models P \quad\mbox{and}\quad \omega \models Q \\
      \omega \models P \lor Q & \mbox{iff} & \omega \models P \quad\mbox{or}\quad \omega \models Q \\
      \omega \models P \arrow Q & \mbox{iff} & \omega \models P \quad\mbox{implies}\quad \omega \models Q \\
      \omega \models \lnot P & \mbox{iff} & \omega \not\models P \\
      \omega \models P \leftrightarrow Q & \mbox{iff} & \omega \models P \quad\mbox{iff}\quad \omega \models Q
    \end{array}
  \]
  \caption{Semantics of Formulas}
  \label{fig:forms}
\end{figure}

% \clearpage
% \bibliographystyle{plainnat}
% \bibliography{bibliography}

\end{document}
