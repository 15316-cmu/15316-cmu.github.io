\documentclass[11pt]{article}

\usepackage[margin=1.5in]{geometry}
\usepackage{lecnotes}
% \usepackage{mpass}
% \lstset{style=verb,language=mpass}
% logic macros, to be filled from
% lmacros-f21.tex (15-814, F21) and lmacros-f16 (15816, F16)

\newcommand{\ddd}{\raisebox{0.2em}[1.1em]{$\vdots$}}
\newcommand{\vctr}[1]{\begin{array}[c]{c}#1\end{array}}
\newcommand{\CC}{\mathcal{C}}
\newcommand{\DD}{\mathcal{D}}
\newcommand{\EE}{\mathcal{E}}
\newcommand{\FF}{\mathcal{F}}

\newcommand{\ms}[1]{\mathsf{#1}}
\newcommand{\mi}[1]{\mathit{#1}}
\newcommand{\mb}[1]{\mathbf{#1}}
\newcommand{\mt}[1]{\mathtt{#1}}
\newcommand{\ums}[1]{\underline{\ms{#1}}}
\newcommand{\oms}[1]{\overline{\ms{#1}}}
\newcommand{\mc}[1]{\mathcal{#1}}

\newcommand{\uscore}{\mbox{\tt\char`\_}}
\newcommand{\arrow}{\mathbin{\rightarrow}}

\newcommand{\semi}{\mathrel{;}}

% from stmaryrd
\newcommand{\lbb}{\llbracket}
\newcommand{\rbb}{\rrbracket}

% evaluation
\newcommand{\eval}{\ms{eval}}
\newcommand{\evalz}{\ms{eval}_{\mathbb{Z}}}
\newcommand{\evalb}{\ms{eval}_{\mathbb{B}}}

% authorization logic
\newcommand{\says}{\mathrel{\mb{says}}}
\newcommand{\aff}{\mathrel{\mb{aff}}}
\newcommand{\llet}[3]{\mb{let}\; #1 = #2\; \mb{in}\; #3}
\newcommand{\alet}[4]{\mb{let}\; \{#1\}_{#2} = #3\; \mb{in}\; #4}

% \newcommand{\rep}[1]{{}^\ulcorner\! #1{}^\urcorner}

\newcommand{\new}[1]{{\color{blue}#1}}





\newcommand{\course}{15-316: Software Foundations of Security \& Privacy}
\newcommand{\lecturer}{Matt Fredrikson}
\newcommand{\lecdate}{January 29, 2026}
\newcommand{\lecnum}{6}
\newcommand{\lectitle}{Memory Safety}
\newcommand{\courseurl}{https://15316-cmu.github.io/2026/}

\begin{document}

\maketitle

\section{Introduction}

The classic \emph{buffer overflow attack} (about which you learn in 15-213
\emph{Computer Systems}) allows a program to take control of your machine while
otherwise innocuous code is executed.  It exploits that accessing memory that
hasn't been explicitly allocated by a program is undefined behavior in C and
therefore \emph{unsafe}.

In this lecture we define a simplistic model of memory and introduce memory
write and read operations into our language.  We then define unsafe behavior and
investigate how to prove safety of programs accessing memory.
For the most part, we see that the reasoning principles involved are quite
similar to how we have done normal assignment to variables, thanks to a
theory of array-valued objects that allows us to treat memory updates as
construction of new array-valued terms. As a result, the only explicit change in
state that occurs is an update to whichever variable maps to the array.

% What can we do if we cannot prove safety, but we'd still like to run a program?
% One option is to rewrite the program by checking that any memory access is in
% bounds before running it.  We show how this can be done for our language, and
% prove safety for the resulting program.  This is one of techniques used in
% \emph{sandboxing} which refers to running untrusted programs in a manner that
% prevents them from doing damage (``\emph{inside a sandbox}'').  There are
% commercial tools such as Intel's
% \href{https://www.intel.com/content/www/us/en/developer/articles/tool/pin-a-dynamic-binary-instrumentation-tool.html}{Pin}
% that can instrument binary code for Intel instruction set architectures.

% \section{A Generic Unsafe Command}

% Before moving on to memory, let's think of safety in more general terms.
% In the example of division, unsafe behavior comes down to a particular
% operation. In general, though, it may be the combination of some operations that
% makes a program unsafe.  For example, a program should not be able to write to
% an output stream after it has been closed.  So it is not the output operation
% \emph{per se}, but a condition associated with it.  Or we may not be able to
% read from an input stream if we are not authorized to do so.  We can capture
% such conditions more generically with the command
% \[
%   \mb{assert}\; P
% \]
% where $P$ is a formula that may depend on variables.  $\mb{assert}\; P$ is
% \emph{unsafe} if $P$ is false; otherwise it is safe but does not change
% the state.
% \[
%   \begin{array}{lcl}
%     \omega \lbb \mb{assert}\; P\rbb \nu & \mbox{iff} & \omega \models P \ \mbox{and}\ \nu = \omega
%     \\[1em]
%     \omega \lbb \mb{assert}\; P\rbb \lightning & \mbox{iff} & \omega \not\models P
%   \end{array}
% \]
% It should be clear that we preserve the property that unsafe programs have no
% poststate.

% Among other things, we could rewrite our programs using $\mb{assert}$ commands.
% For example, if we replaced $x := \mb{divide}\; e_1\; e_2$ by
% $\mb{assert}\; \lnot (e_2 = 0) \semi x := e_1 / e_2$ the two programs would have
% the same meaning in every state (either both unsafe, or both safe and
% determinate).

% How do we reason about $[\mb{assert}\; P]Q$?  If $P$ is true, then the
% postcondition $Q$ must be true.  If $P$ is false, then $Q$ is irrelevant: the
% formula $[\mb{assert}\; P]Q$ is always false.  These two conditions are neatly
% captured by the axiom
% \[ 
%   [\mb{assert}\; P]Q \leftrightarrow P \land Q
% \]
% Here are the corresponding right and left rules of the sequent calculus.
% \begin{rules}
%   \infer[{[\mb{assert}]R}]
%   {\Gamma \vdash [\mb{assert}\; P]Q, \Delta}
%   {\Gamma \vdash P, \Delta
%     & \Gamma \vdash Q, \Delta}
%   \hspace{3em}
%   \infer[{[\mb{assert}]L}]
%   {\Gamma, [\mb{assert}\; P]Q \vdash \Delta}
%   {\Gamma, P, Q \vdash \Delta}
% \end{rules}
% Let's prove that the axiom is actually valid.  From that, the soundness of the
% rules as previously explained for the axiom $[\semi]A$ for sequential program
% composition.

% \begin{theorem}
%   The axiom
%   \[ 
%     [\mb{assert}\; P]Q \leftrightarrow P \land Q
%   \]
%   is valid.
% \end{theorem}
% \begin{proof}
%   We start with the proof from right to left.  We set up,
%   for an arbitrary state $\omega$:
%   \begin{tabbing}
%     $\omega \models P \land Q$ \` (assumption) \\
%     $\ldots$ \\
%     $\omega \models [\mb{assert}\; P]Q$ \` (to show)
%   \end{tabbing}
%   In order to show the conclusion, we have to show
%   two properties: (a) if $\omega\lbb \mb{assert}\; P\rbb \nu$
%   then $\nu \models Q$, and (b) \textbf{not} $\omega \lbb \mb{assert}\; P\rbb \lightning$.

%   (a) follows, since $\nu = \omega$ by definition of $\lbb {-}\rbb$
%   and $\omega \models Q$ from our assumption.

%   (b) follows by definition of $\lightning$ since $\omega \models P$ from
%   our assumption.

%   \medskip
%   \noindent For the proof from left to right, we set up
%   \begin{tabbing}
%     $\omega \models [\mb{assert}\; P]Q$ \` (assumption) \\
%     $\ldots$ \\
%     $\omega \models P \land Q$ \` (to show)
%   \end{tabbing}
%   By the definition of $\models$, the assumption gives us (a) for every $\nu$
%   with $\omega \lbb \mb{assert}\; P\rbb \nu$ we have $\nu \models Q$, and (b)
%   \textbf{not} $\omega \lbb \mb{assert}\; P \rbb \lightning$.

%   From (b) we know $\omega \models P$ (otherwise $\mb{assert}\; P$ would be
%   unsafe).  Therefore, by definition of $\lbb {-}\rbb$ and our assumption we
%   have $\omega = \nu$, so $\omega \models Q$.

%   Taking these two together we have $\omega \models P \land Q$.
% \end{proof}

\section{A Proof of Safety}

Before diving in to talk about the semantics of memory and memory safety,
we start with a brief follow-up to our coverage of safety from the previous lecture.
Proofs of safety can often be significantly simpler than proof of correctness.
On the other hand, sometimes safety depends critically on some other correctness
property.

We reconsider the example from \href{\courseurl/lectures/05-safety.pdf}{Lecture 5}.
\[
  x \geq 6 \arrow [\mb{while}\; (x > 1)\; x := x-2]\, 0 \leq x \leq 1
\]
To prove this, we required a loop invariant, and $x \geq 0$ was sufficient.

We can modify this to introduce a division, and just prove safety (so the
postcondition is $\top$.
\[
  x \geq 6 \land y = x \arrow [\mb{while}\; (x > 2)\; \{x := x-1 \semi y := y-1 \semi z := \mb{divide}\; z\; y\}]\, \top 
\]
After one step (${\arrow}R$) it remains to prove
\[
  x \geq 6, y = x \vdash [\mb{while}\; (x > 2)\; \{x := x-1 \semi y := y-1 \semi z := \mb{divide}\; z\; y\}]\, \top 
\]
Here, we pick the weakest loop invariant we can think of, namely $J = (y = x)$.
Then we have to prove:
\begin{description}
\item[True Initially:]  $x \geq 6, y = x \vdash y = x$, which is manifestly valid.
\item[Preserved:] We lose the antecedent $x \geq 6$ and $y=x$, but we add the
  loop invariant and the loop guard.  So we have to show
  \[
    y = x, x > 2 \vdash [\{x := x-1 \semi y := y-1 \semi z := \mb{divide}\; z\; y\}](y = x)
  \]
  Using the rules $[\mb{:=}]R$ and $[\mb{divide}]R$, this reduces to showing
  \[
    y = x, x > 2, x' = x-1, y' = y-1 \vdash \lnot (y' = 0)
  \]
  and
  \[
    y = x, x > 2, x' = x-1, y' = y-1, z' = z / y' \vdash y' = x'
  \]
  Both of these are manifestly valid.
\item[Implies Postcondition:] Again, without the antecedent, but this time
  with the negated loop guard, we have to prove the postcondition $\top$.
  So:
  \[
    y = x, \lnot (x > 2) \vdash \top
  \]
  Again, this is easily seen to be true.
\end{description}
So to prove safety, we need a nontrivial loop invariant in this example
(which corresponds to proving a simple element of correctness as part of safety).

On the other hand, safety is still easy for the following modified program:
\[
  x \geq 6 \vdash [\mb{while}\; (x > 1)\; \{y := \mb{divide}\; y\; x \semi x := x-2\}]\, \top 
\]
By the loop guard we see that $x > 1$ inside the loop body, so the division is
safe.  However, if we had written $\mb{divide}\; x\; y$ then there is an
immediate counterexample with $y = 0$.


\subsection{A Theorem about Safety\protect\footnotemark}
\footnotetext{mentioned, but not explicitly stated in lecture}
\label{sec:safety-thm}

\begin{theorem}[Soundness of Dynamic Logic with Unsafe Programs]
  All the rules of the sequent calculus are sound, and all the axioms we stated
  are valid.
\end{theorem}
\begin{proof}
  By considering each case and reasoning along similar lines as in the sample
  proofs of such properties in lecture.
\end{proof}

We can rigorously state that if we can prove \emph{some} postcondition for
$\alpha$ then $\alpha$ is safe.  The theorem assumes that we have proved the
soundness of all the sequent calculus rules (or axioms) we use in the formal
proof (as claimed in the preceding theorem).

\begin{theorem}[Safety]
  If $\cdot \vdash P \arrow [\alpha]Q$ then there is no $\omega$ with
  $\omega \models P$ such that $\omega \lbb \alpha\rbb \lightning$.
\end{theorem}
\begin{proof}
  Assume $\cdot \vdash P \arrow [\alpha]Q$ and for some
  $\omega$ we have $\omega \models P$ and $\omega \lbb \alpha \rbb \lightning$.
  We have to show a contradiction.

  By soundness of the sequent calculus we have
  $\omega \models P \arrow [\alpha]Q$.  Since $\omega \models P$ we obtain
  $\omega \models [\alpha]Q$.  By definition, this implies that \textbf{not}
  $\omega \lbb \alpha\rbb \lightning$, which is a contradiction.
\end{proof}

\section{Writing and Reading Memory}

Mathematically, we model memory as a map from $\mathbb{Z}$ (the index domain) to
$\mathbb{Z}$ (the value domain).  It is total, but may be \emph{indeterminate}
on some indices.  In the programming language we assume that each memory value
comes with an allocated size, and accessing memory outside its bounds is
\emph{unsafe}.

As explained in \href{\courseurl/lectures/05-safety.pdf}{Lecture 5}, we would
like to keep expressions \emph{safe}, but possibly \emph{indeterminate}.  Unsafe
behavior is then exhibited only by commands and the programs constructed from
them.  Sticking to this approach, we add a new kind of variable, $M$, that
stands for memory (along with $N, \ldots$), and the new commands
\begin{itemize}
\item $M := \mb{alloc}\; e$ \quad allocate a fresh memory block of size $e$ and
  store it in $M$.  This is unsafe if the value of $e$ is negative.
\item $M[e_1] := e_2$ \quad write $e_2$ into memory $M$ at address $e_1$.
  This is unsafe if the value of $e_1$ is out of bounds for $M$.
\item $x := M[e]$ \quad set $x$ to the contents of memory at address $e$ into
  $x$.  This is unsafe if the value of $e$ is out of bounds for $M$.
\end{itemize}
The design decision that memory access takes place via commands means that you
have to rewrite hypothetical code such as
\[
  M[x] := (M[x-1] + M[x+1])/2
\]
in the more verbose form
\[
  \begin{array}{l}
    t_1 := M[x-1] \semi \\
    t_2 := M[x+1] \semi \\
    t_3 := \mb{divide}\; (t_1 + t_2)\; 2 \semi \\
    M[x] := t_3
  \end{array}
\]

Next, we need to rigorously define the semantics of the new commands so that (a)
we can implement them, and (b) we can prove soundness of our axioms and rules to
reason about them (including their safety).

The first issue is how to track the contents of memory.  For that purpose, we
change our definition of the state $\omega$.  So far the state has been a total
function from variables to integers, $\omega : \ms{Var} \arrow \mathbb{Z}$ where
$\ms{Var}$ is the (countably infinite) set of variables.  Now variables can also
map to memory.
\[
  \omega : \ms{Var} \arrow (\mathbb{Z} \cup (\mathbb{Z} \arrow \mathbb{Z}))
\]
We assume that programs map lowercase variables to integers and uppercase
variables to memory, so that there is never any confusion between the two forms.
Mathematically, we use the letter $H : \mathbb{Z} \arrow \mathbb{Z}$ (suggesting
a \emph{heap}).  In addition, each heap comes with an allocated size $|H|$.
All indices $0 \leq i < |H|$ are in bounds; all other indices are out of bounds.
All our previous semantic definitions remain unchanged since all variables in
those definitions stand for integers.

We write $|H|$ for the allocated size of a heap $H$.  This is a term and it is
always determinate.

We also use $|M|$ as an expression.  If $\omega \lbb M\rbb = H$ then
\[
  \omega \lbb |M|\rbb = |H|
\]

We begin with safe and unsafe memory reads.
\[
  \begin{array}{lcl}
    \omega \lbb x := M[e]\rbb \nu
    & \mbox{iff}
    & \omega \lbb e\rbb = i \ \mbox{and}\ \omega \lbb M\rbb = H\ \mbox{and}\
      \nu = \omega[x \mapsto H(i)] \\
    & & \quad \mbox{provided $0 \leq i < |H|$}
    \\[1em]
    \omega \lbb x := M[e]\rbb \lightning
    & \mbox{iff}
    & \omega \lbb e\rbb = i \ \mbox{and}\ \omega \lbb M\rbb = H\ \mbox{and \textbf{not}}\ 0 \leq i < |H|
  \end{array}
\]
It should be clear that unsafe programs continue to satisfy no postcondition.

Memory write follows the same intuition, we just have to make sure to suitably
update the map defining the state of memory.
\[
  \begin{array}{lcl}
    \omega \lbb M[e_1] := e_2\rbb \nu
    & \mbox{iff}
    & \omega \lbb e_1\rbb = i \ \mbox{and}\ \omega \lbb e_2\rbb = a\ \mbox{and}\ 
      \omega \lbb M\rbb = H\ \mbox{and} \\
    & & \quad H' = H[i \mapsto a] \ \mbox{and}\ \nu = \omega[M \mapsto H'] \\
    & & \quad \mbox{provided $0 \leq i < |H|$}
    \\[1em]
    \omega \lbb M[e_1] := e_2\rbb \lightning
    & \mbox{iff}
    & \omega \lbb e_1\rbb = i \ \mbox{and}\ \omega \lbb M\rbb = H\ \mbox{and \textbf{not}}\ 0 \leq i < |H|
  \end{array}
\]

Finally, allocation creates a fresh heap of the requested size.  We do not
model the initial contents, so the only information we obtain about the newly
allocated heap $H$ is its size.
\[
  \begin{array}{lcl}
    \omega \lbb M := \mb{alloc}\; e\rbb \nu
    & \mbox{iff}
    & \omega \lbb e\rbb = n\ \mbox{and}\ n \geq 0\ \mbox{and}\ \nu = \omega[M \mapsto H] \\
    & & \quad \mbox{for some $H$ with $|H| = n$}
    \\[1em]
    \omega \lbb M := \mb{alloc}\; e\rbb \lightning
    & \mbox{iff}
    & \omega \lbb e\rbb = n\ \mbox{and}\ n < 0
  \end{array}
\]

Before moving on, it is worth taking note that we explicitly do not allow
assigning one heap variable to another, i.e., we do not consider
$N := M$ to be a valid program. If we did, we would need to model this
semantically as copying the contents of $M$ into a new heap $N$.
This can be accomplished by iterating over the length of $M$ and copying
each memory cell into $N$.

\section{Reasoning about Memory}

In order to reason about memory we need to introduce expressions that capture
what we know about the state of memory.  Mathematically, this leads us to the
\emph{theory of arrays} \citep{McCarthy62ifip} which we can view as being
constructed on top of the theory of arithmetic we have assumed so far.  Because
of the importance of arrays in imperative programming, some efficient decision
procedures have been devised (see, for example, \citet{Stump01lics}) and
implemented in provers such as Z3 or the CVC family.

In the theory of arrays we have two expressions $\mb{read}\; H\; i$ and
$\mb{write}\; H\; i\; a$ where $H$ denotes an array, $i$ and index into an
array, and $a$ a value stored in the array.  Note that the expression
$\mb{write}\; H\; i\; a$ denotes a ``new'' array; semantically $H[i \mapsto a]$.
For us, both the index domain and the values are integers.

To reason about safety of memory access we also need to talk about the allocated
size of a heap.  Mathematically, we write $|H|$ for this size.  Writes preserve
the allocated size, so we have the axiom
\[
  |\mb{write}\; H\; i\; a| = |H|
\]

There are two axioms, called \emph{read over write}, that allow us to
reason about these expressions.
\[
  \begin{array}{l}
    i = k \rightarrow \mb{read}\; (\mb{write}\; H\; i\; a)\; k = a \\
    i \neq k \rightarrow \mb{read}\; (\mb{write}\; H\; i\; a)\; k = \mb{read}\; H\; k
  \end{array}
\]
In addition we have an \emph{axiom of extensionality} which states that two
arrays are equal if they agree on all elements.  In our use of the theory
of arrays, the quantifier ranges over integers.
\[
  (\forall i. H(i) = H'(i)) \land |H| = |H'| \rightarrow H = H'
\]
As usual, we treat the new expressions as always denoting either an array or an
integer, although the value may sometimes be indeterminate.

The right rules of the sequent calculus for these new commands are now
relatively straightforward.
\begin{rules}
  \infer[{[\mb{read}]R^{x'}}]
  {\Gamma \vdash [x := M[e]]Q(x), \Delta}
  {\Gamma \vdash 0 \leq e < |M|, \Delta
    & \Gamma, x' = \mb{read}\; M\; e \vdash Q(x'), \Delta}
\end{rules}
As for assignment, the $x'$ must be chosen fresh in $[\mb{read}]R^{x'}$.  The
same is true for $M'$ in the following rules.
\begin{rules}
  \infer[{[\mb{write}]R^{M'}}]
  {\Gamma \vdash [M[e_1] := e_2]Q(M), \Delta}
  {\Gamma \vdash 0 \leq e_1 < |M|, \Delta
    & \Gamma, M' = \mb{write}\; M\; e_1\; e_2 \vdash Q(M'), \Delta}
\end{rules}
For allocation, the main thing that we need to be careful about is breaking
any association with the newly-allocated heap and the previous heap's contents.
This is accomplished by creating a fresh variable $M'$ and introducing a
single assumption about its size.
\begin{rules}
  \infer[{[\mb{alloc}]R^{M'}}]
  {\Gamma \vdash [M := \mb{alloc}\; e]Q(M), \Delta}
  {\Gamma \vdash 0 \leq e, \Delta
    & \Gamma, |M'| = e \vdash Q(M'), \Delta}
\end{rules}

The aspect of these rules critical for safety is the first premise that requires
us to prove safety (independently of the postcondition).

\subsection{Uninitialized Memory}

A student in class asked what happens if we apply these rules in a situation
where an array is freshly allocated, and then read from prior to any writes
taking place. This is a good opportunity to see them in action.

Supposing we have a program $M := \mb{alloc}\; 5; x := M[0]$, and we want to
query whether $0 < x$ afterwards. We start by setting this up in a sequent,
and removing the sequential composition.
\begin{rules}
  \infer[{[;]R}]
  {\cdot \vdash [M := \mb{alloc}\; 5; x := M[0]]0 < x}
  {\infer[{[\mb{alloc}]R^{M'}}]
    {\cdot \vdash [M := \mb{alloc}\; 5][x := M[0]]0 < x}
    {\cdot \vdash 0 \leq 5
      & |M'| = 5 \vdash [x := M'[0]]0 < x}}
\end{rules}
The left premise is true by arithmetic. Continuing with the right premise,
\begin{rules}
  \infer[{[\mb{read}]R^{x'}}]
  {|M'| = 5 \vdash [x := M'[0]]0 < x}
  {|M'| = 5 \vdash 0 \le 0 < |M'|
    & |M'| = 5, x' = \mb{read}\; M'\; 0 \vdash 0 < x'}
\end{rules}
Again the left premise follows immediately by arithmetic, but the right
premise is stuck. This answers the original question: just like with
uninitialized variables, we simply can't prove anything about the
uninitialized contents of memory. Any attempt to prove non-vacuous
properties involving uninitialized memory will at some point get stuck
as we see in this example.

\section{A Small Example of Memory Safety}
\label{sec:mem-example}

Consider the following program to initialize memory up to $n$:
\[
  i := 0 \semi \mb{while}\; (i < n) \{\, M[i] := i \semi i := i+1\, \}
\]
This is patently unsafe: just consider a state where $|M| = k$ and
$n = k+1$.  Then the last time around the loop we will have $i = k$, leading to
an unsafe memory access at $M[k]$.

We can add a precondition $n \leq |M|$ and then try to
prove safety with
\[
  n \leq |M| \arrow [i := 0 \semi \mb{while}\; (i < n) \{\, M[i] := i \semi i := i+1\,\}]\top
\]
We could try $i \leq n$ but that's insufficient.  Here is what preservation
would require:
\[
  i \leq n, i < n \vdash [M[i] := i \semi i := i+1] i \leq n
\]
We note two problems: (1) safety will fail because we cannot prove that $0 \leq i$
and (2) we have lost the precondition $n \leq |M|$ so we also cannot conclude
that $i < |M|$.

Let's try a more complex invariant:
\[
  0 \leq i \leq n \leq |M|
\]
Now preservation requires
\[
  0 \leq i \leq n \leq |M|, i < n \vdash [M[i] := i \semi i := i+1]\, (0 \leq i \leq n \leq |M|)
\]
This reduces in two steps to
\[
  0 \leq i \leq n \leq |M|, i < n, M' = \mb{write}\; M\; i\; i, i' = i+1 \vdash 0 \leq i' \leq n \leq |M'|
\]
Fortunately, this is valid (even with the useless assumption about $M'$).
Because our postcondition is just $\top$, that is easily seen to be true, but
the loop invariant does not hold initially because
\[
  n \leq |M| \vdash 0 \leq 0 \leq n \leq |M|
\]
is not valid (counterexample: $n = -1$).  So we need to strengthen our precondition
to
\[
  0 \leq n \leq |M|
\]

% \section{Guards}

% Consider the scenario where you are given the program from the previous section
% but no loop invariant.  We might be able to guess a loop invariant, but if not
% we are stuck.  The program looks unsafe, even with the precondition
% $0 \leq n \leq |M|$.  If we still need to run it, what can we do?  One option
% would be \emph{dynamic monitoring}: we track memory accesses as the program
% executes and abort it if it attempts to do something unsafe.  Another one is to
% \emph{instrument it with guards} before memory accesses.  These guards abort the
% program if the access would be unsafe and let it go on if they are safe.
% Aborting programs is considered safe, because aborting is actually a
% well-defined operation that does no harm (except to the running program, but it
% is its own fault if it tries to execute an unsafe command).  For this purpose we
% need a new command $\mb{test}\; P$.  In the literature on dynamic logic this is
% called a \emph{guard} and written as ${?} P$.  It has the following
% specification.
% \[
%   \begin{array}{lcl}
%     \omega \lbb \mb{test}\; P\rbb \nu & \mbox{iff} & \omega \models P \ \mbox{and}\ \nu = \omega
%     \\[1ex]
%     \omega \lbb \mb{test}\; P\rbb \lightning & \mbox{iff} & \mbox{false}
%   \end{array}
% \]
% The program $\mb{test}\; \bot$ will not have a poststate, but it is also safe
% because it aborts.  As a result, based on the definition of $\omega \models [\alpha]Q$
% it is the case that
% \[
%   \omega \models [\mb{test}\; \bot] Q
% \]
% for any $\omega$ and $Q$.  This in turn means that $[\mb{test}\; \bot]Q$ is
% logically valid and $\cdot \vdash [\mb{test}\; \bot]Q$ should be derivable.

% Just to be sure, let's recall the definition of $\omega \models [\alpha]Q$
% from \href{\courseurl/lectures/05-safety.pdf}{Lecture 5}, page L5.4.
% \[
%   \begin{array}{lcl}
%     \omega \models [\alpha]Q
%     & \mbox{iff}
%     & \mbox{for every $\nu$ with $\omega \lbb \alpha \rbb \nu$ we have $\nu \models Q$} \\
%     & & \quad \mbox{and \textbf{not} $\omega \lbb \alpha \rbb \lightning$}
%   \end{array}
% \]
% This is a \emph{partial correctness} statement: if there is no poststate $\nu$
% such that $\omega \lbb \omega\rbb \nu$, then the first part of the condition is
% vacuously true.

% What does this mean for the axiom for $[\mb{test}\; P]Q$?  If $P$ is true, then
% $Q$ should also be true.  But if the test succeeds then we know $P$, so we
% conjecture (somewhat rashly, perhaps)
% \[
%   [\mb{test}\; P]Q \leftrightarrow (P \arrow Q)
% \]
% What if $P$ is false?  Then the program $\mb{test}\; P$ has no poststate, and
% yet it is safe.  Consequently $[\mb{test}\; P]Q$ should be true, and by this
% axiom it will be becase $\bot \arrow Q$ is valid.

% Let's prove that this axiom is valid, just as in Theorem 1 of
% \href{\courseurl/lectures/05-safety.pdf}{Lecture 5} we proved that
% $[\mb{assert}\; P]Q \leftrightarrow (P \land Q)$.

% \begin{theorem}
%   The axiom
%   \[ 
%     [\mb{test}\; P]Q \leftrightarrow (P \arrow Q)
%   \]
%   is valid.
% \end{theorem}
% \begin{proof}
%   From right to left we set up for an arbitrary $\omega$
%   \begin{tabbing}
%     $\omega \models P \arrow Q$ \` (assumption) \\
%     $\ldots$ \\
%     $\omega \models [\mb{test}\; P]Q$ \` (to show)
%   \end{tabbing}
%   The conclusion holds if for every $\nu$ such that
%   $\omega \lbb \mb{test}\; P\rbb \nu$ we have $\nu \models Q$ (and \textbf{not}
%   $\omega \models \lbb \mb{test}\; P\rbb \lightning$, which is true).
  
%   So we assume $\omega \lbb \mb{test}\; P\rbb \nu$ and have to show that
%   $\nu \models Q$.  By definition, this second assumption give us
%   $\omega \models P$ and $\nu = \omega$.
  
%   By the first assumption also $\omega \models Q$ and since $\nu = \omega$
%   we have $\nu \models Q$

%   \vspace{1em}\noindent
%   For the left to right direction, we set up for an arbitrary $\omega$
%   \begin{tabbing}
%     $\omega \models [\mb{test}\; P]Q$ \` (assumption) \\
%     $\ldots$ \\
%     $\omega \models P \arrow Q$ \` (to show)
%   \end{tabbing}
%   So we assume $\omega \models P$ and it remains to show that
%   $\omega \models Q$.  Since $\omega \models P$, the first assumption gives us
%   $\nu \models Q$ for any $\nu$ with $\omega \lbb \mb{test}\; P\rbb \nu$.  We
%   can use this for $\nu = \omega$ (since $\omega \models P$) to obtain
%   $\omega \models Q$.
% \end{proof}

% We can easily turn the two directions of the axiom into right and
% left rules of the sequent calculus.
% \begin{rules}
%   \infer[{[\mb{test}]R}]
%   {\Gamma \vdash [\mb{test}\; P]Q, \Delta}
%   {\Gamma, P \vdash Q, \Delta}
%   \hspace{3em}
%   \infer[{[\mb{test}]L}]
%   {\Gamma, [\mb{test}\; P]Q \vdash \Delta}
%   {\Gamma \vdash P, \Delta
%     & \Gamma, Q \vdash \Delta}
% \end{rules}

% Here is a little table on the differences between $\mb{assert}\; P$ and
% $\mb{test}\; P$.

% \vspace{1em}
% \[
%   \begin{tabular}{c|c|c}
%     Poststate & $\omega \lbb \mb{assert}\; P\rbb \nu$ & $\omega \lbb \mb{test}\; P\rbb \nu$ \\
%     & iff $\omega \models P$ and $\nu = \omega $ & iff $\omega \models P$ and $\nu = \omega$
%     \\[1ex]\hline & & \\[-1ex]
%     Safety & $\omega \lbb \mb{assert}\; P\rbb \lightning$ & $\omega \lbb \mb{test}\; P\rbb \lightning$ \\
%     & iff $\omega \not\models P$ & never
%     \\[1ex]\hline & & \\[-1ex]
%     Axiom & $[\mb{assert}\; P]Q \leftrightarrow P \land Q$ & $[\mb{test}\; P]Q \leftrightarrow (P \arrow Q)$
%   \end{tabular}
% \]

% \section{Sandboxing}

% Sandboxing unsafe behavior (including memory access through the read or write
% commands) proceeds as follows.  We replace
% \begin{itemize}
% \item every allocation $M := \mb{alloc}\; e$ with the program
%   $\mb{test}\; 0 \leq e \semi M := \mb{alloc}\; e$
% \item every memory read $x := M[e]$ with the program
%   $\mb{test}\; 0 \leq e < |M| \semi x := M[e]$
% \item every memory write $M[e_1] := e_2$ with the program
%   $\mb{test}\; 0 \leq e_1 < |M| \semi M[e_1] := e_2$
% \item every division $x := \mb{divides}\; e_1\; e_2$ with
%   the program $\mb{test}\; e_2 \neq 0 \semi x := \mb{divides}\; e_1\; e_2$.
% \item every assertion $\mb{assert}\; P$ with
%   the program $\mb{test}\; P$
% \end{itemize}
% Now we can safely execute the program.  Equally importantly, perhaps, we can
% prove the safety of the program transformed in this manner.

% \begin{theorem}[Safety of Sandboxed Programs]
%   Given a program $\alpha$ under precondition $P$, we obtain the sandboxed
%   $\alpha'$ as defined in the preceding paragraph.
  
%   Then $\cdot \vdash P \arrow [\alpha']\top$.
% \end{theorem}
% \begin{proof}
%   We prove safety using the loop invariant $\top$ for every loop.  Since any
%   potentially unsafe command is immediately preceded by a guard, the safety
%   condition incorporated into the rule will be provable since it is exactly the
%   assumption enabled by the postcondition of the guard.

%   More formally, this proof would be carried out by an \emph{induction over the
%     structure of formulas and programs}.
% \end{proof}

% There are two optimizations that come to mind.  We can introduce fresh
% temporaries in order to avoid recomputing the value of expressions.  For
% example, instead of $\mb{test}\; 0 \leq e < |M| \semi x := M[e]$ we would
% insert $t := e \semi \mb{test}\; 0 \leq t < |M| \semi x := M[t]$ for a fresh
% temporary variable $t$.

% The other optimization is a bit trickier.  At first one might think that if we
% can \emph{prove} $P$ when we encounter $\mb{test}\; P$ during the verification
% of safety we can replace it by $\mb{skip}$ (or $\mb{assert}\; \top$, which
% should be equivalent).  However, in conditionals the postcondition is
% replicated:
% \[
%   [\mb{if}\; P\; \mb{then}\; \alpha\; \mb{else}\; \beta]Q
%   \leftrightarrow
%   (P \arrow [\alpha]Q) \land (\lnot P \arrow [\beta]Q)
% \]
% When the postcondition contains a program (which arises from sequential
% composition, for example), this program may be proved twice, once in each
% branch.  A similar remark applies if we unfold loops because the program
% $\alpha$ is replicated.

% So we can only replace $\mb{test}\; P$ with $\mb{skip}$ only if for all
% occurrences of $[\mb{test}\; P]Q$ in a safety proof we can prove $P$.

% Returning to our earlier example, we can sandbox
% \[
%   n \leq |M| \arrow [i := 0 \semi \mb{while}\; (i < n) \{\, M[i] := i \semi i := i+1\,\}]\top 
% \]
% as
% \[
%   n \leq |M| \arrow [i := 0 \semi \mb{while}\; (i < n) \{\, 
%   \mb{test}\; 0 \leq i < |M| \semi M[i] := i \semi i := i+1\,\}]\top  
% \]
% Without a loop invariant and stronger precondition we can't eliminate the guard.
% With the additional information from \autoref{sec:mem-example} it would be
% redundant and can be dropped.

\section{Summary}

Since it has been a while, we summarize the language so far.  We restrict
programs from containing certain expressions with indeterminate behavior to
retain the property that for every given prestate $\omega$, every program has
three possible outcomes: a poststate $\nu$, or no poststate in which case it may
be safe or unsafe ($\lightning$).
\[
  \begin{array}{llcll}
    \mbox{Variables} & x, y, z \\
    \mbox{Memory} & M, N, \ldots \\[1ex]
    \mbox{Constants} & c & ::= & \ldots, -1, 0, 1, \ldots \\[1ex]
    \mbox{Expressions} & e & ::= & c \mid x \mid e_1 + e_2 \mid e_1 * e_2 \mid e_1 - e_2 & \mbox{determinate} \\
                     & & \mid & |M| & \mbox{determinate} \\
                     & & \mid & e_1 / e_2 \mid \mb{read}\; M\; e \mid \mb{write}\; M\; e_1\; e_2
                         & \mbox{may be indeterminate} \\[1ex]
    \mbox{Programs} & \alpha, \beta & ::= & x := e \mid \alpha \semi \beta \mid \mb{skip} \\
                     & & \mid & \mb{if}\; P\; \mb{then}\; \alpha\; \mb{else}\; \beta
                                \mid \mb{while}\; P\; \alpha \\
                     & & \mid & \mb{test}\; P & \mbox{safe} \\
                     & & \mid & \mb{assert}\; P \mid x := \mb{divide}\; e_1 \; e_2 & \mbox{may be unsafe} \\
                     & & \mid & M := \mb{alloc}\; e & \mbox{may be unsafe} \\
                     & & \mid & x := M[e] \mid M[e_1] := e_2 & \mbox{may be unsafe} \\[1ex]
    \mbox{Formulas} & P, Q & ::= & e_1 \leq e_2 \mid e_1 = e_2 \mid \top \mid \bot \\
                     & & \mid  & P \land Q \mid P \lor Q \mid P \arrow Q \mid P \leftrightarrow Q \mid
                                 \lnot P \\
                     & & \mid & \forall x.\, P(x) \mid \exists .\, P(x) \\
                     & & \mid & [\alpha]Q \mid \langle \alpha\rangle Q
  \end{array}
\]

\bibliographystyle{plainnat}
\bibliography{../../2024/lectures/bibliography}

\end{document}
