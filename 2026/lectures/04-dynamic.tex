\documentclass[11pt]{article}

\usepackage[margin=1.5in]{geometry}
\usepackage{lecnotes}
% \usepackage{mpass}
% \lstset{style=verb,language=mpass}
% logic macros, to be filled from
% lmacros-f21.tex (15-814, F21) and lmacros-f16 (15816, F16)

\newcommand{\ddd}{\raisebox{0.2em}[1.1em]{$\vdots$}}
\newcommand{\vctr}[1]{\begin{array}[c]{c}#1\end{array}}
\newcommand{\CC}{\mathcal{C}}
\newcommand{\DD}{\mathcal{D}}
\newcommand{\EE}{\mathcal{E}}
\newcommand{\FF}{\mathcal{F}}

\newcommand{\ms}[1]{\mathsf{#1}}
\newcommand{\mi}[1]{\mathit{#1}}
\newcommand{\mb}[1]{\mathbf{#1}}
\newcommand{\mt}[1]{\mathtt{#1}}
\newcommand{\ums}[1]{\underline{\ms{#1}}}
\newcommand{\oms}[1]{\overline{\ms{#1}}}
\newcommand{\mc}[1]{\mathcal{#1}}

\newcommand{\uscore}{\mbox{\tt\char`\_}}
\newcommand{\arrow}{\mathbin{\rightarrow}}

\newcommand{\semi}{\mathrel{;}}

% from stmaryrd
\newcommand{\lbb}{\llbracket}
\newcommand{\rbb}{\rrbracket}

% evaluation
\newcommand{\eval}{\ms{eval}}
\newcommand{\evalz}{\ms{eval}_{\mathbb{Z}}}
\newcommand{\evalb}{\ms{eval}_{\mathbb{B}}}

% authorization logic
\newcommand{\says}{\mathrel{\mb{says}}}
\newcommand{\aff}{\mathrel{\mb{aff}}}
\newcommand{\llet}[3]{\mb{let}\; #1 = #2\; \mb{in}\; #3}
\newcommand{\alet}[4]{\mb{let}\; \{#1\}_{#2} = #3\; \mb{in}\; #4}

% \newcommand{\rep}[1]{{}^\ulcorner\! #1{}^\urcorner}

\newcommand{\new}[1]{{\color{blue}#1}}





\newcommand{\course}{15-316: Software Foundations of Security \& Privacy}
\newcommand{\lecturer}{Matt Fredrikson}
\newcommand{\lecdate}{January 21, 2024}
\newcommand{\lecnum}{4}
\newcommand{\lectitle}{Dynamic Logic}
\newcommand{\courseurl}{https://15316-cmu.github.io/2024/}

\begin{document}

\maketitle

\section{Introduction}

In the last lecture we introduced \emph{propositional sequent calculus} because
it is the foundation of most of the calculi we will investigate and because it
helps us understand the basic principles underlying the sequent calculus.  Let's
summarize them:
\begin{itemize}
\item We define \emph{sequents} $\Gamma \vdash \Delta$ with antecedents
  (assumptions) and succedents (goals).
\item We give a \emph{meaning} to sequents (also called a \emph{semantics}): a
  sequent is \emph{valid} if when all antecedents are true then some succedent
  is true.
\item We also give \emph{inference rules} to prove sequents formally in a
  bottom-up manner.  These are divided into \emph{right rules} (how to prove a
  succedent) and \emph{left rules} (how to use an antecedent), and the identity
  rule connecting left and right.
\item We connect the inference rules to the semantics by proving
  (mathematically, at the metalevel) that the sequent calculus is \emph{sound}
  and \emph{complete}:
  \begin{itemize}
  \item A rule is sound if the validity of all premises implies
    the validity of the conclusion.  If all rules are sound, the
    whole system of inference rules is sound.
  \item A system of rules is \emph{complete} if every valid sequent
    can be proved with them.
  \end{itemize}
\end{itemize}
In the specific case of propositional sequent calculus we were able to prove
completeness using the following steps:
\begin{itemize}
\item Every rule is \emph{invertible} in the sense that if the conclusion is
  valid then all the premise must also be valid.
\item Every rule is \emph{reductive} in the sense that all premises are smaller
  than the conclusion by counting the number of connectives and logical
  constants.
	\end{itemize}
	As we move forward, we will have to give up some of these properties while
	retaining others.

	In this lecture we begin the study of \emph{dynamic logic}, which enriches the
	sequent calculus with formulas that talk directly about program execution.  The
	key new ingredients are the modalities $[\alpha]Q$ and $\langle \alpha\rangle
	Q$, which express (partial) correctness properties of a program $\alpha$ with
	respect to a postcondition $Q$.  We will define the syntax of our tiny language,
	give semantics to these modal formulas, and then derive proof rules for common
	program constructs such as assignments and conditionals.

\section{Dynamic Logic: A Logic with Programs}

In this course we use a tiny imperative programming language so we can be
rigorous about the concepts we introduce.  With it we illustrate and analyze the
concepts that transfer to realistic languages.  There will be small differences
regarding the precise extent of the language as we move through various
concepts.  Here is our first cut.  \emph{Expressions} denote integers and are
either constants, variables, or operators like addition and multiplication.
Programs include variable assignment, sequential composition, conditionals, and
loops.  Formulas no longer have propositional variables (for simplicity), but we
add comparisons between integers to the usual set of logical constants and
connectives.  New here are two kinds of formulas, $[\alpha]Q$ and
$\langle \alpha\rangle Q$ that mention programs.  We explain them below the
table.
\[
  \begin{array}{llcl}
    \mbox{Variables} & x, y, z \\
    \mbox{Constants} & c & ::= & \ldots, -1, 0, 1, \ldots \\
    \mbox{Expressions} & e & ::= & c \mid x \mid e_1 + e_2 \mid e_1 * e_2 \mid \ldots \\
    \mbox{Programs} & \alpha, \beta & ::= & x := e \mid \alpha \semi \beta
                                            \mid \mb{if}\; P\; \mb{then}\; \alpha\; \mb{else}\; \beta
                                            \mid \mb{while}\; P\; \alpha \\
    \mbox{Formulas} & P, Q & ::= & e_1 \leq e_2 \mid e_1 = e_2 \mid \ldots \\
                     & & \mid  & P \land Q \mid P \lor Q \mid P \arrow Q \mid P \leftrightarrow Q \mid
                                 \lnot P \mid \top \mid \bot \\
                     & & \mid & \forall x . P(x) \mid \exists x . P(x) \\
                     & & \mid & [\alpha]Q \mid \langle \alpha\rangle Q
  \end{array}
\]
A \emph{state} is a total map from variables to integer values.  We use
$\omega, \mu, \nu$ for states.  A program represents a partial function from an
initial state (called \emph{prestate}) to a final state (called
\emph{poststate}).  It is a partial function because loops may not terminate, so
no final state may every be reached.  The sequence of states that a program goes
through is its \emph{trace} as discussed above.

Characteristic of \emph{dynamic logic}~\citep{Harel79book,Harel00book} are two
modalities that mention programs:
\begin{itemize}
\item $[\alpha]Q$ (pronounced ``\emph{box alpha $Q$}'') which is true if,
  starting in a prestate $\omega$, the formula $Q$ will be true in every
  poststate $\nu$ we can reach by executing program $\alpha$.
\item $\langle \alpha\rangle Q$ (pronounced ``\emph{diamond alpha $Q$}'') which
  is true in a state $\omega$ if there is a poststate $\nu$ that we can
  reach by executing $\alpha$ in which $Q$ is true.
\end{itemize}
We refer to $Q$ in these formulas as a \emph{postcondition}.  These definitions
are formulated to account for \emph{nondeterministic} programs that may have
multiple poststates for a given prestate.  In our case of a deterministic
language, this will be zero or one.  Therefore, we can already see that
$\langle \alpha\rangle Q$ should imply $[\alpha]Q$.

In the next lecture we introduce a rigorous \emph{semantics} for dynamic logic
(which, by necessity, also includes programs and expressions).  For today, we
instead try to derive some rules keeping in mind the informal semantics and our
understanding how an imperative programming language executes.

\section{Semantics of Formulas}

The semantics of formulas in a given state must appeal to the meaning of
expressions and the meaning of programs.  Therefore the meanings of programs and
formulas mutually depend on each other.  We start with some simple cases.
\[
  \begin{array}{lcl}
    \omega \models e_1 \leq e_2 & \mbox{iff} & \omega \lbb e_1\rbb \leq \omega \lbb e_2\rbb \\
    \omega \models e_1 = e_2 & \mbox{iff} & \omega \lbb e_1\rbb = \omega \lbb e_2\rbb \\[1ex]
    \omega \models P \land Q & \mbox{iff} & \omega \models P \quad\mbox{and}\quad \omega \models Q \\
    \omega \models P \lor Q & \mbox{iff} & \omega \models P \quad\mbox{or}\quad \omega \models Q \\
    \omega \models P \arrow Q & \mbox{iff} & \omega \models P \quad\mbox{implies}\quad \omega \models Q \\
    \omega \models \lnot P & \mbox{iff} & \omega \not\models P \\
    \omega \models P \leftrightarrow Q & \mbox{iff} & \omega \models P \quad\mbox{iff}\quad \omega \models Q \\
    \omega \models \forall x.\, P(x) & \mbox{iff} & \omega[x \mapsto c] \models P(x) \quad \mbox{for every $c \in \mathbb{Z}$} \\
    \omega \models \exists x.\, P(x) & \mbox{iff} & \omega[x \mapsto c] \models P(x) \quad \mbox{for some $c \in \mathbb{Z}$}
  \end{array}
\]
Here $\omega[x \mapsto c]$ denotes the state obtained from $\omega$ by mapping $x$
to $c$ and leaving all other variables unchanged.

For programs, we have to recall the informal definition from earlier.  
$[\alpha]Q$ is true if $Q$ is true in \emph{every} poststate of
$\alpha$.  Because a nonterminating program does not have a poststate, this is a
statement about the \emph{partial correctness} of the program $\alpha$.
Conversely, $\langle \alpha\rangle Q$ is true if $Q$ is true in \emph{some}
poststate of $\alpha$.
\[
  \begin{array}{lcl}
    \omega \models [\alpha]Q & \mbox{iff}
    & \mbox{for every $\nu$ with $\omega \lbb\alpha\rbb \nu$
      we have $\nu \models Q$} \\
    \omega \models \langle\alpha\rangle Q & \mbox{iff}
    & \mbox{there is a $\nu$ with $\omega \lbb \alpha \rbb \nu$
      and $\nu \models Q$}
  \end{array}
\]

Now we say $P$ is \emph{valid} (written as $\models P$) if $\omega \models P$
for every state $\omega$.  (Recall that all states are defined on all variables,
so this is well-defined.)

A sequent $P_1, \ldots, P_n \vdash Q_1, \ldots, Q_m$ is valid if for every state
$\omega$, whenever for all antecedents $P_i$ we have $\omega \models P_i$ then
for some succedent $Q_j$ we have $\omega \models Q_j$.

\section{Rules and Axioms}

Our goal is ultimately to find proof rules for Dynamic Logic formulas containing
boxes and diamonds. We'll start with a general observation: if $P$ and $Q$ are
equivalent (i.e. $P \leftrightarrow Q$ is valid), then rules such as
\begin{rules}
  \infer[]
  {\Gamma \vdash P, \Delta}
  {\Gamma \vdash Q, \Delta}
  \hspace{3em}
  \infer[]
  {\Gamma, P \vdash \Delta}
  {\Gamma, Q \vdash \Delta}
\end{rules}
are \textbf{both sound and invertible}.  That's because $P$ and $Q$ are always
either both false or both true, regardless of the state because the
bi-implication $P \leftrightarrow Q$ is valid. We call these valid formulas
\emph{axioms}, and we'll use this observation to identify axioms that 
we can use to derive sound and invertible proof rules.

In order to obtain good left and right rules from axioms we just
have to make sure the rules are reductive.  For example, the following
equivalence is reductive in that it has the effect of making the program 
$\alpha\,;\,\beta$ smaller, breaking it into $\alpha$ and $\beta$ that
reside in separate boxes.
\[
  [\alpha \semi \beta]Q \leftrightarrow [\alpha]([\beta]Q)
\]
Let's prove this.  We start with the right-to-left direction, which implies the
\emph{soundness} of $[\semi]R$ and \emph{invertibility} of $[\semi]L$.
We set up the proof:
\begin{tabbing}
  $\omega \models [\alpha]([\beta]Q)$ \` (assumption) \\
  $\ldots$ \\
  $\omega \models [\alpha \semi \beta]Q$ \` (to show)
\end{tabbing}
We now walk through the proof in individual steps, narrowing the gap, sometimes
from below and sometimes from above.  Typically, one only presents the end
result and the reader has to figure out how one might have obtained it.

By definition, the conclusion holds if for every state $\nu$ such that
$\omega \lbb \alpha \semi \beta\rbb \nu$ we have $\nu \models Q$.  Now our proof
state is (highlighting the new parts in \new{blue}):
\begin{tabbing}
  $\omega \models [\alpha]([\beta]Q)$ \` (1, assumption) \\
  \new{$\omega\lbb \alpha \semi \beta\rbb \nu$ for some $\nu$} \` \new{(2, assumption)} \\
  \new{$\ldots$} \\
  \new{$\nu \models Q$} \` \new{(to show)} \\
  $\omega \models [\alpha \semi \beta]Q$ \` \new{(by defn.\ of $\models$)}
\end{tabbing}
By definition, assumption 2 is true if there is some intermediate state $\mu$
such that $\omega \lbb \alpha \rbb \mu$ and $\mu \lbb \beta \rbb \nu$.
Let's write this into the proof as well.
\begin{tabbing}
  $\omega \models [\alpha]([\beta]Q)$ \` (1, assumption) \\
  $\omega\lbb \alpha \semi \beta\rbb \nu$ for some $\nu$ \` (2, assumption) \\
  \new{$\omega\lbb \alpha\rbb \mu$ and $\mu \lbb \beta \rbb \nu$ for some $\mu$}
  \` \new{(3, from 2 by defn.\ of $\lbb{-}\rbb$)} \\
  $\ldots $ \\
  $\nu \models Q$ \` (to show) \\
  $\omega \models [\alpha \semi \beta]Q$ \` (by defn.\ of $\models$)
\end{tabbing}
Next: from assumption 1 and the fact that $\omega \lbb \alpha \rbb \mu$
we can conclude that $\mu \models [\beta]Q$, again just by the definition
of $\models$.
\begin{tabbing}
  $\omega \models [\alpha]([\beta]Q)$ \` (1, assumption) \\
  $\omega\lbb \alpha \semi \beta\rbb \nu$ for some $\nu$ \` (2, assumption) \\
  $\omega\lbb \alpha\rbb \mu$ and $\mu \lbb \beta \rbb \nu$ for some $\mu$
  \` (3, from 2 by defn.\ of $\lbb{-}\rbb$) \\
  \new{$\mu \models [\beta]Q$} \` \new{(4, from 1 and 3(a) by defn.\ of $\models$)} \\
  $\ldots $ \\
  $\nu \models Q$ \` (to show) \\
  $\omega \models [\alpha \semi \beta]Q$ \` (by defn.\ of $\models$)
\end{tabbing}
Now we use the same argument knowing that $\mu \lbb \beta \rbb \nu$
and $\mu \models [\beta]Q$ to conclude that $\nu \models Q$.  But that's
what we needed to show!
\begin{tabbing}
  $\omega \models [\alpha]([\beta]Q)$ \` (1, assumption) \\
  $\omega\lbb \alpha \semi \beta\rbb \nu$ for some $\nu$ \` (2, assumption) \\
  $\omega\lbb \alpha\rbb \mu$ and $\mu \lbb \beta \rbb \nu$ for some $\mu$
  \` (3, by defn.\ of $\lbb{-}\rbb$ from 2) \\
  $\mu \models [\beta]Q$ \` (4, from 1 and 3(a) by defn. of $\models$) \\
  $\nu \models Q$ \` \new{(5, from 4 and 3(b) by defn. of $\models$)} \\
  $\omega \models [\alpha \semi \beta]Q$ \` (\new{from 5 and 2} by defn.\ of $\models$)
\end{tabbing}
We see the proof is actually quite straightforward.  We just have to
carefully unwind the definitions.

Here is the proof in the other direction.\footnote{not shown in lecture}  We set up:
\begin{tabbing}
  $\omega \models [\alpha \semi \beta]Q$ \` (1, assumption) \\
  $\ldots$ \\
  $\omega \models [\alpha]([\beta]Q)$ \` (to show)
\end{tabbing}
We show the filled-in proof.  You can probably walk through it in the order
we made the deductions.
\begin{tabbing}
  $\omega \models [\alpha \semi \beta]Q$ \` (1, assumption) \\
  $\omega \lbb \alpha\rbb \mu$ for some $\mu$ \` (2, assumption) \\
  $\mu \lbb \beta\rbb \nu$ for some $\nu$ \` (3, assumption) \\
  $\omega \lbb \alpha \semi \beta\rbb \nu$ \` (4, from 2 and 3 by defn.\ of $\lbb {-}\rbb$ \\
  $\nu \models Q$ \` (5, from 1 and 4 by defn of $\models$) \\
  $\mu \models [\beta]Q$ \` (6, from 5 and 3 by defn of $\models$) \\
  $\omega \models [\alpha]([\beta]Q)$ \` (7, from 6 and 2 by defn of $\models$)
\end{tabbing}

Now that we know that $[\alpha \semi \beta]Q \leftrightarrow [\alpha]([\beta]Q)$ is
valid, we can derive left and right rules for reasoning about sequential composition.
\begin{rules}
  \infer[{[\semi]}R]
  {\Gamma \vdash [\alpha \semi \beta]Q, \Delta}
  {\Gamma \vdash [\alpha]([\beta]Q), \Delta}
  \quad
  \infer[{[\semi]}L]
  {\Gamma, [\alpha \semi \beta]Q \vdash \Delta}
  {\Gamma, [\alpha]([\beta]Q) \vdash \Delta}
\end{rules}

\section{Some Axioms for Dynamic Logic}

Based on the insights from the previous section and the rules for programs in
dynamic logic, we can conjecture the following axioms.  And, indeed, they are
all valid, even though we don't show the proofs.  We write the postfix ``$A$''
to indicate that what we are naming is not a rule but an axiom.
\[
  \begin{array}{lll}
    [:=]A & [x := e]Q(x) \leftrightarrow \forall x'. x' = e \arrow Q(x') \quad
            \mbox{($x'$ not in $e$ or $Q(x)$)} \\\relax
    [\semi]A & [\alpha \semi \beta]Q \leftrightarrow [\alpha]([\beta]Q) \\\relax
    [\mb{if}]A & [\mb{if}\; P\; \mb{then}\; \alpha\; \mb{else}\; \beta]Q
                  \leftrightarrow (P \arrow [\alpha]Q) \land (\lnot P \arrow [\beta]Q) \\\relax
    [\mb{unfold}]A & [\mb{while}\; P\; \alpha] \leftrightarrow
                     (P \arrow [\alpha][\mb{while}\; P\; \alpha]Q)
                     \land (\lnot P \arrow Q)
  \end{array}
\]
Because these axioms are valid biconditionals, we obtain correct left and right
rules for the sequent calculus. Note that directly translating these axioms into
rules may be somewhat unsatisfactory. For example, the rules for $[\mb{unfold}]$ would
not be reductive, and the rules for $[:=]$ and $[\mathbf{if}]$ introduce complicated
formulas. In the next lecture, we'll see how this can be addressed.

For now, let's take a closer look at the $[:=]A$ axiom, which may not seem obvious
at first glance. The first instinct might be the following axiom for assignment
\[
  [x \leftarrow e]P \leftrightarrow (x = e \rightarrow P) \qquad (\mbox{WRONG})
\]
However, this is \emph{not valid} and could therefore lead to unsound
reasoning. For example, the following is certainly not
valid
\[
  \not\models x = 3 \rightarrow [x \leftarrow x+1] (x = 17)
\]
since computing $x \leftarrow x+1$ will set $x$ to $4$ but the
postcondition requires $x$ to be $17$.  With the wrong axiom we could
prove
\[
  (x = 3 \rightarrow [x \leftarrow x+1] x = 17) \leftrightarrow
  (x = 3 \rightarrow ((x = x+1) \rightarrow x = 17))
\]
and the right-hand side is true since $x = x+1$ is contradictory.

There are two ways out: one is to \emph{carefully} substitute $e$ for
$x$ with a so-called \emph{uniform substitution}.  The other is to
\emph{rename} the variable $x$, something we also did when generating
a verification condition for a loop.  This is handled by
quantification over a fresh variable that does not occur in $P$.  We
write $P(x)$ for a formula $P$ with (possible) occurrences of $x$, and
then $P(x')$ for the result of renaming all occurrences of $x$ to
$x'$.  Then our axiom becomes
\[
  [x \leftarrow e]P(x) \leftrightarrow (\forall x'.\, x' = e \rightarrow P(x'))
\]
It is important for soundness that $x'$ is a variable that does not
already occur in $e$ or $P(x)$.  We often refer to this as a ``fresh
variable''.

	Our example no longer gives us a contradiction, because
	\[
	  (x = 3 \rightarrow [x \leftarrow x+1] x = 17) \leftrightarrow
	  (x = 3 \rightarrow \forall x'.\, x' = x+1 \rightarrow x' = 17)
	\]
	is false as it should be.

		\section{Addendum: Soundness of the Assignment Axiom}

		In this addendum we prove that the assignment axiom $[:=]A$ is valid.
		Recall the statement (with $x'$ \emph{fresh}, i.e.\ $x'$ does not occur in $e$ or $Q(x)$):
		\[
		  [x := e]Q(x) \leftrightarrow \forall x'.\, x' = e \arrow Q(x')
		\]

		We start with the left-to-right direction.  We set up the proof:
		\begin{tabbing}
		  $\omega \models [x := e]Q(x)$ \` (1, assumption) \\
		  $\ldots$ \\
		  $\omega \models \forall x'.\, x' = e \arrow Q(x')$ \` (to show)
		\end{tabbing}
    Now we proceed like before.
		\begin{tabbing}
		  $\omega \models [x := e]Q(x)$ \` (1, assumption) \\
		  $\omega \lbb x := e \rbb \nu$ for $\nu = \omega[x \mapsto \omega\lbb e\rbb]$ \` (2, by defn. of $\lbb :=\rbb$) \\
      $\omega[x \mapsto \omega\lbb e \rbb] \models Q(x)$ \` (3, by 2 and defn of $[\cdot]$) \\
		  $\omega[x' \mapsto \omega\lbb e \rbb] \models Q(x')$ for fresh $x'$ \` (4, by 3 and $x'$ fresh) \\
		  $\omega[x' \mapsto c] \models x' = e \rightarrow Q(x')$ for every $c \in \mathbb{Z}$  \` (5, by 4 and defn. $\rightarrow$) \\
		  $\omega \models \forall x'.\, x' = e \arrow Q(x')$ \` (by defn.\ of $\forall$)
		\end{tabbing}

		In the above, $Q(x')$ is the result of renaming the free occurrences of $x$ in $Q$ to the fresh variable $x'$.
		Since $c = \omega\lbb e\rbb$, the value of $x$ in $\nu$ equals the value of $x'$ in $\omega[x' \mapsto c]$,
		and all other variables agree with $\omega$; therefore $\nu \models Q(x)$ implies $\omega[x' \mapsto c] \models Q(x')$.
		This is exactly what we needed, completing the left-to-right direction.

		Here is the proof in the other direction. 
		\begin{tabbing}
		  $\omega \models \forall x'.\, x' = e \arrow Q(x')$ \` (1, assumption) \\
      $\omega[x' \mapsto c] \models x' = e \arrow Q(x')$ for all $c \in \mathbb{Z}$ \` (2, by defn. $\forall$) \\
      $\omega[x' \mapsto \omega\lbb e\rbb] \models Q(x')$ \` (3, by 2 and defn. of $\rightarrow$) \\
      $\omega[x \mapsto \omega\lbb e\rbb] \models Q(x)$ \` (4, by 3 and $x'$ fresh) \\
		  $\omega \lbb x := e \rbb \nu$ for any $\nu = \omega[x \mapsto \omega\lbb e\rbb]$ \` (5, by defn. of $\lbb :=\rbb$) \\
		  $\omega \models [x := e]Q(x)$ \` (6, from 4 and 5 by defn.\ of $[\cdot]$)
		\end{tabbing}

		\bibliographystyle{plainnat}
		\bibliography{bibliography}
	
	\end{document}
